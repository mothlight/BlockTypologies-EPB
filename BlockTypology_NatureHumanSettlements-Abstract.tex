In the urban age, where the majority of the world's population live in cities, it is critical we improve our understanding of the strengths and limitations of existing city designs to ensure they are safe, clean, can deliver health co-benefits and importantly, are sustainable into the future. To enable this, a systematic and efficient means of performing inter- and intra-city comparisons based on urban form is urgently required. Until now, methods for comparing cities have been limited by scalability, often reliant upon non-standardised local input data that can be costly and difficult to obtain. To address this, we have developed a unique approach to determine the mix, distribution, and composition of neighbourhood types in cities based on dimensions of block size and regularity, sorted by a self-organising map. We illustrate the utility of the method to provide an understanding of the underlying city morphology by overlaying spatially standardised city metrics such as air pollution and transport activity across a set of 1667 global cities with populations exceeding 300,000. The approach reports associations between specific mixes of neighbourhood typologies and quantities of moving vehicles (r=0.97), impervious surfaces (r=0.86), and air pollution levels (aerosol optical depth r=0.58 and NO$_{2}$ r=0.57). This approach can identify the characteristics and neighbourhood mixes of well-performing urban areas while also producing city `fingerprints' that can be used to provide new metrics, insights, and drive improvements in city design now and into the future.

