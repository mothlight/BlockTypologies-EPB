\section{Introduction}

Poorly managed urbanisation can impact urban residents through increased air pollution\cite{Stevenson2016,Sallis2016,Landrigan2017}, urban heat\cite{Coutts2012,Bowler2010}, urban sprawl\cite{Frank2000,Bettencourt2010}, and social isolation\cite{Vlahov2002}. In contrast, well managed urbanisation can promote economic activities and innovation, increase employment opportunities with economies of scale reducing the need for sprawling infrastructure\cite{Kuhnert2006,Bettencourt2007,Lobo2013}. For sustainable future cities, managing urbanisation processes so that they minimise the negative externalities and maximise the liveability, prosperity, and resilience of a city requires data and knowledge about how cities work. What is urgently needed therefore, is a systematic approach to compare cities that is a systematic method to identify different neighbourhoods within a city and to perform intra-city comparisons\cite{Louf2014a} to provide insights into i) the performance of cities at precinct or neighbourhood scales with respect to elements such as air pollution, infrastructure and urban heat and ii) what urban morphologies are best suited to the growing urban challenges of the 21st Century. To address this need, our framework allows us to compare a single element, as small as neighbourhoods of cities, across the globe, and to discover the mix and spatial distribution of different neighbourhood types in cities. We refer to this as an individual city `fingerprint'.

To discover how different cities work, research has been conducted through a number of methods to categorise cities and analyse underlying processes. Building on recent advances in computing power, artificial intelligence, and the wide availability of urban imagery, approaches have been created to discover unique characteristics of cities and how cities function. Large numbers of geo-tagged photos have been used to detect patterns of urban usage and public perception of a number of areas' functional and social attributes\cite{Liu2016,Zhou2014a}. Place Pulse, a database of urban imagery using crowd-sourced classifications (including safety, beauty, and liveliness) has attempted to quantify perceptions of urban areas\cite{Dubey2016,Naik2014} and inequality\cite{Salesses2013}. Doersch\cite{Doersch2012} used a large number of geo-localised street level images to discover common visual features across a number of cities. Enabled by remote sensing data, night-time light data has been used to categorise cities into stages of urbanisation and levels of economic activities\cite{Zhang2013}. Urban characteristics (road geometry, building dimensions and heights, and vegetation heights) have also been used to classify cities into typologies of differing periods of historical design and urban planning (i.e. 19th Century, 1950s, 1970s, etc.)\cite{Hermosilla2014}. 

Analysing the structural elements of road networks and urban blocks, generally the longest lasting part of an urban area, can provide clues as to the processes under which city development occurred (and currently continues)\cite{Porta2006a,Strano2012}. Road infrastructure can point to the dominant modes of transportation and governance systems underlying each urban area, with grid structures reflecting a top-down planning system\cite{Crouch1977,Courtat2011}, while T-shaped crossings point to more disorderly\cite{Jacobs1961} self-organised organic growth\cite{Cardillo2006}. Division of large land blocks (often originally agricultural land) can follow an evolutionary progression, either to medium sized manufacturing or smaller residential plots\cite{Fialkowski2008}. In addition, studies show that areas unconstrained by adjoining villages or topography are generally and most efficiently subdivided into smaller grids (i.e. regular rectangular plots)\cite{Strano2012}. The connection between the physical and topological structure of the road network in cities has parallels to the structural sociology field and transportation and economics. The `space syntax' community and Hillier\cite{Hillier1996} established a correlation between configurations of urban forms and variations of human interactions within it. 

Most of the methods described above require some amount of subjective classification of local input data; the quality and availability of which can vary widely when attempting to apply these methods to all types of cities world-wide, across multiple political districts. Existing empirical methods highlight underlying urban development mechanisms by evaluating the street network typology\cite{Hillier1989}, but this approach neglects their geometrical expanse and their function as places to stay. A method cannot rely solely on topology but needs to incorporate the urban geometry\cite{Louf2014a}.

Our unique method uses neighbourhood-level block size and regularity information to comprehensively explore the characteristics of cities across selected domains. The fundamental nature of city blocks, defined as the area bounded by surrounding streets, can be read as a simplified schematic view of the city\cite{Southworth2013}, highlighting both the structure and organisation of the city, as well as the process of the urban development and morphology. This makes the city block the most used and accessible urban element for urban analysis and basic common elements in city design typologies and theory\cite{Jacobs1961,Ewing2010,Louf2014a}. Blending theory with globally available datasets at scale can enable new insights into the form and function of the world's cities. This method allows us to find the mix of commonalities and unique elements, a `fingerprint' of each city, to understand what works and doesn't work in a city, an understanding essential to best manage the world's rapid urbanisation for health and well-being.



